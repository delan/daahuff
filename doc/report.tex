\documentclass[a4paper,12pt]{article}
\usepackage[utf8]{inputenc}
\usepackage[T1]{fontenc}
\usepackage[margin=1in]{geometry}
\usepackage{parskip}
\usepackage{graphicx}
\usepackage{hyperref}
\usepackage{listings}
\usepackage{multirow}
\usepackage[usenames,dvipsnames]{color}
\hypersetup{
	colorlinks,
	pdfauthor=Delan Azabani,
	pdftitle=Design and Analysis of Algorithms 300: Huffman assignment
}
\lstset{basicstyle=\ttfamily, basewidth=0.5em}

\title{Design and Analysis of Algorithms 300\\Huffman assignment}
\date{April 26, 2014}
\author{Delan Azabani}

\begin{document}

\maketitle

\section{Implemented functionality}

\begin{itemize}
	\item Generating optimal frequency tables
	\item Compression with symbols of a single \texttt{char}
	\item Text interpretation from and to a radix 64 encoding
	\item Decompression with symbols of arbitrary length
\end{itemize}

\section{Assumptions, defects and limitations}

When generating a Huffman tree from a frequency table with a single symbol, the
leaf node will become the head and only node. With no movements required in
tree traversal, the only symbol's Huffman code will be empty. Compression
yields nothing, and decompression results in following a null reference.

Instead of implementing a special case for this, for example by using a false
Huffman tree with a head and one child, I have checked for and disallowed
compression and decompression with a frequency table having just one symbol.

Decompression may yield a string that is up to five characters too long, with
added garbage on the end. This is due to the nature of radix 64 text encoding,
which requires the input to be padded with zero bits up to a multiple of six
bits.

To resolve ambiguities with the frequency table format, the line feed, carriage
return and backslash are represented in the table by the C-style escape
sequences \texttt{\textbackslash n}, \texttt{\textbackslash r} and
\texttt{\textbackslash\textbackslash} respectively. However, I did not
implement a way to allow for symbols containing these characters
\emph{among} other characters.

\section{Build information}

This assignment was developed using Visual C\# 2013, but it compiles and
runs correctly on Visual C\# 2010 without any changes. To run it, open
\texttt{Asgn.sln} and hit F5.

\end{document}
