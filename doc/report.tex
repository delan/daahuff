\documentclass[a4paper,12pt]{article}
\usepackage[utf8]{inputenc}
\usepackage[T1]{fontenc}
\usepackage[margin=1in]{geometry}
\usepackage{parskip}
\usepackage{graphicx}
\usepackage{hyperref}
\usepackage{listings}
\usepackage{multirow}
\usepackage[usenames,dvipsnames]{color}
\hypersetup{
	colorlinks,
	pdfauthor=Delan Azabani,
	pdftitle=Design and Analysis of Algorithms 300: Huffman assignment
}
\lstset{basicstyle=\ttfamily, basewidth=0.5em}

\title{Design and Analysis of Algorithms 300\\Huffman assignment}
\date{April 25, 2014}
\author{Delan Azabani}

\begin{document}

\maketitle

\section{Implemented functionality}

\begin{itemize}
	\item Generating optimal frequency tables
	\item Compression with symbols of a single \texttt{char}
	\item Text interpretation from and to base 64
	\item Decompression with symbols of arbitrary length
	\item Appropriate exceptions thrown and handled
\end{itemize}

\section{Known defects}

When generating a Huffman tree from a single symbol, the leaf node will become
the head and only node, and thus the symbol's Huffman code will be empty. I
have checked and disallowed compression and decompression with a frequency
table having one symbol.

Decompression may yield a string that is a few characters too long or too
short.

\section{Handling newlines and tabs}

To resolve ambiguities with the frequency table format, the line feed, carriage
return, horizontal tab and backslash characters are represented in the table by
the C-style escape sequences
\texttt{\textbackslash n},
\texttt{\textbackslash r},
\texttt{\textbackslash t} and
\texttt{\textbackslash\textbackslash} respectively.

\section{Build information}

This implementation is based on \texttt{DAA300Asgn\_Base.zip}. While it was
developed using Visual C\# 2013, it compiles and runs correctly on Visual C\#
2010 without changes. To run this assignment, open \texttt{Asgn.sln} and hit F5
within Visual Studio.

\end{document}
